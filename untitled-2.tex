\documentclass[a4paper]{article}
\usepackage[slovene]{babel}
\usepackage[T1]{fontenc}
\usepackage[utf8]{inputenc}
\usepackage{lmodern}
\usepackage{amsfonts}
\usepackage{amsmath}
\usepackage{makeidx}
\usepackage{siunitx}
\usepackage{graphicx}
\usepackage{subfigure}
\usepackage{amsfonts}
\usepackage{amssymb}


\title{FINANČNI PRAKTIKUM \\\vspace{3cm} {\huge k-total rainbow domination numer vs. domination number}\vspace{3cm}}
\author{Tim Resnik \\[1.5mm] Lana Herman \\[1.5mm]\vspace{7cm}}
\date{November, 2019}

\begin{document}

\begin{titlepage}
\clearpage \maketitle
\thispagestyle{empty}
\end{titlepage} 

\maketitle
\section{Uvod}
\text{V najini projektni nalogi se bova ukvarjala s trditvijo, ki pravi, da za graf G in} $k \geq 4$ \text{obstaja tesna povezava} $\gamma_{krt}(G) \geq 2\gamma(G)$. Torej opisuje povezavo med k-rainbow t.d.n. in d.n. V programu \textit{Sage} bova za majhne grafe izračunala koeficient $\frac{\gamma_{krt}(G)}{\gamma(G)}$ in poskusila najti rezultat, ki bo manjši od 2. Če  bi nama to uspelo, bi prišla do protiprimera in naloga bi se končala. Če nama do protiprimera ne uspe priti, potem se bova usmerila na večje grafe. Generizirala bova naključen graf z $n \geq 15$ vozlišči. Nato bova z odstranjevanjem in dodajanjem povezav iskala tak graf, ki bo imel zgoraj omenjen koeficient manjši od 2.

\end{document}