\documentclass[a4paper]{article}
\usepackage[slovene]{babel}
\usepackage[T1]{fontenc}
\usepackage[utf8]{inputenc}
\usepackage{lmodern}
\usepackage{amsfonts}
\usepackage{amsmath}
\usepackage{makeidx}
\usepackage{graphicx}
\graphicspath{ {./images/} }



\title{Finančni praktikum \\\vspace{2cm} {\huge $k$-total rainbow domination numer vs. domination number}\vspace{2cm}}
\author{Tim Resnik \\[1.5mm] Lana Herman \\[1.5mm]\vspace{6cm}
Univerza v Ljubljani \\[1.5mm]
Fakulteta za matematiko in fiziko \vspace{2cm}}
\date{November, 2019}

\begin{document}

\begin{titlepage}
\clearpage \maketitle
\thispagestyle{empty}
\end{titlepage}

\tableofcontents
\pagebreak

\section{Problem naloge}

V projektni nalogi se bova ukvarjala z domeno, ki se ukvarja s povezavo med 'k-rainbow total domination number' (označimo z $\gamma_{krt}(G)$)  in 'domination number' (označimo z $\gamma(G)$). Definiciji za $\gamma_{krt}(G)$ in $\gamma(G)$ sta v razdelku \textbf{Razlaga pojmov}.\\
Domneva pravi, da za graf G in $k \geq 4$ \text{obstaja tesna povezava} $\gamma_{krt}(G) \geq 2\gamma(G)$. Cilj najine projetkne naloge je najti tak graf, za katerega ta neenkost ne drži. To sva na majhnih grafih preverila na konkretnih primerih, kjer sva število vozlišč in število $k$ vnesla ročno. Za večje grafe sva uporabila metodo \textit{Simulated Annealing}.\\
Poiskala sva tudi primere, za katere velja enakost $\gamma_{krt}(G) = 2\gamma(G)$.

\section{Razlaga pojmov}

Graf $G$ ima množico vozlišč $V(G)$ in množico povezav $E(G)$. Za množico $N_G(v)$ velja, da vsebuje vsa sosednja vozlišča $v$, v grafu G. Za grafa G in H, je kartezični produkt $G \square H$ graf z množico vozlišč $V(G) \times V(H)$.\\
\textit{Dominirana množica} grafa $G$ je $D \subseteq V(G)$, taka da za vsako vozlišče $v \in V(G)$ in $v \notin D$ velja, da je sosed nekemu vozlišču iz $D$. \textit{Dominirano število}, $\gamma(G)$, je velikost najmanjše dominirane množice. Če za $\forall v \in V(G)$ velja, da je sosed vozlišču iz $D$, za $D$ rečemo, da je \textit{totalno dominirana množica} grafa $G$. \textit{Totalno dominirano število}, $\gamma_{t}(G)$, je velikost najmanjše totalno dominirane množice.\\
Za pozitivno celo število $k$, je \textit{'k-rainbow domination function'} ($k$RDF) grafa $G$ funkcija $f$, ki slika iz $V(G)$ v množico $\{1, \cdots, k\}$. Zanjo velja, da za katerikoli $v \in V(G)$ in $f(v) = \emptyset$ velja $\cup_{u \in N_G(v)} f(u) = [k]$. Definiramo $\|f\| = \sum_{v \in V(G)}|f(v)|$. $\|f\|$ rečemo \textit{teža} $f$-a. \textit{'k-rainbow domination number'}, $\gamma_{kr}(G)$, grafa $G$ je minimalna vrednost $\|f\|$ za vse 'k-rainbow domination functions'. Po definiciji vemo, da za vse $k \geq 1$ velja $$\gamma_{kr}(G) = \gamma(G \square K_k).$$
Graf $K_k$ predstavlja polni graf na $k$ vozliščih. Nazadnje definirajmo še \textit{'k-rainbow total domination function'} ($k$RTDF), katera se od 'k-rainbow domination function' razlikuje v dodatnem pogoju, ki zagotavlja, da če za $\forall v \in V(G)$ velja $f(v) = \{i\}$, potem obstaja tak $u \in N_G(v)$, da je $i \in f(u)$. \textit{'k-rainbow total domination number'}, $\gamma_{krt}(G)$, grafa $G$ je minimalna vrednost $\|f\|$ za vse 'k-rainbow total domination functions'. Tudi tu za vse $k \geq 1$ velja $$\gamma_{krt}(G) = \gamma_t(G \square K_k).$$
\pagebreak

\section{Reševanje problema}

\subsection{Majhni grafi}

Za majhne grafe sva definirala naslednjo pseudokodo.
\begin{figure}[h!]
    \centering
    \includegraphics[width=13cm, height=5.35cm]{Slika1}
    \label{fig:mesh1}
\end{figure}\\
V kodi seznam \textit{result} predstavlja seznam vseh koeficientov $\frac{\gamma_{krt}(G)}{\gamma(G)}$, seznam \textit{gresult} predstvalja vse tiste grafe, za katere velja neenakost $\frac{\gamma_{krt}(G)}{\gamma(G)} < 2$, seznam \textit{dvaresult} pa predstavlja grafe za katere velja enakost $\frac{\gamma_{krt}(G)}{\gamma(G)} = 2$.\\
Prva zanka preteče $i$ naključnih grafov, generiranih z že vgrajeno funkcijo iz programa \textit{sage}, imenovano \textit{graphs.RandomGNP(n,p)}, kjer $n$ predstvalja število vozlišč, $p$ pa verjetnost, da posamezno povezavo doda v graf. Ta zanka je porabi $\mathcal{O}(n)$ časa.\\
Druga zanka se zapelje čez določeni število $k$-jev, kjer $k$ predstvalja število vozlišč na polnem grafu. Po definiciji namreč  vemo, da 'k-rainbow total domination number' izračunamo preko kartezičnega produkta med grafom $G$ in polnim grafom s $k$ vozlišči. Druga zanka porabi $\mathcal{O}(m)$ časa, kjer je $m$ število grafov, ki jih preteče $k$.\\
Funkcija \textit{cartesian\_product()} porabi $\mathcal{O}(n m)$ časa, saj vsakemu vozlišču iz grafa $G$ ($n$ vozlišč) priredi vsako vozlišče iz $k$-polnega grafa ($m$ vozlišč).\\
Torej je časovna zahtevnost te pseudokode enaka $\mathcal{O}(n^2 m^2)$.\\[1.5mm]
To kodo sva uporabila za $n \in \{5, \cdots, 20\}$ na različnem številu korakov in različnih $k$-jih. Za $k \geq 3$ sva minimum zmeraj dobila enak 2, za $k = 2$ pa sva dobila veliko protiprimerov, npr.:
\pagebreak
\begin{figure}[h!]
    \centering
    \includegraphics[width=7.5cm, height=6cm]{Slika2}
    \label{fig:mesh1}
\end{figure}\\
Opazimo, da je takih grafov kar 205 (koda je preverila na 1000 grafih s 15 vozlišči). To število se seveda spreminja, ko kodo poženemo večkrat. Zato sva jo pognala večrat in na različnem številu vozlišč. Vse skupaj je koda pregledala približno $10^{100}$ grafov, izključno za $k = 2$. Podatke sva izpisala v program \textit{Matlab}, jih za posamezno število vozlišč povprečila in jih aproksimirala z že vgrajeno funkcijo \textit{interp1}. Dobila sva spodnji graf.
\begin{figure}[h!]
    \centering
    \includegraphics[width=7.5cm, height=6cm]{Slika3}
    \label{fig:mesh1}
\end{figure}\\
Opazimo, da protiprimerov ne dobimo za grafe z manj kot 7 vozlišč, za večje grafe pa velja, da gre število protiprimerov proti nič, ko se veča število vozlišč. Največ pa jih je pri grafih s približno 25 vozlišči.
\pagebreak

\subsection{Veliki grafi}

Za velike grafe sva definirala naslednjo pseudokodo.
\begin{figure}[h!]
    \centering
    \includegraphics[width=13cm, height=11cm]{Slika4}
    \label{fig:mesh1}
\end{figure}\\
Funkcija \textit{spremeni\_graf()} nam ne nepovezan graf tako spremeni, da mu naključno odsrani povezavo in naključno doda povezavo. To funkcijo sva definirala zato, da sva jo lahko uporabila v naslenji kodi. Njena časovna zahtevnost je $\mathcal{O}(n^2)$, saj gre čez dve for zanki, vsaka ima zahtevnost $\mathcal{O}(n)$.\\
Zadnja koda pa predstavlja metodo \textit{Simulated Annealing}. To je metoda za približevanje h globalnemu optimumu dane funkcije. Tako sva s spreminjanjem danega grafa iskala tak graf, za katerega bo veljalo $\frac{\gamma_{krt}(G)}{\gamma(G)} < 2$. To kodo sva pognala na grafih z vsaj 20 vozlišči. Na podlagi najinih poizkusov sva ugotovila, da za grafe s sodim številom vozlišč, za vsak $k>=4$, je koeficient enak 2 , za grafe z lihim številom vozlišč pa velja formula $\frac{n}{\lfloor{\frac{n}{2}}\rfloor}$, v primeru k>=5. Opazimo torej, če pošljemo $n \rightarrow \infty$, gre koeficient proti 2.\\
\pagebreak

Prvi graf ima 20 vozlišč, $k=4$ in $p=0.5$, drugi pa ima $k=5$ in $p=0.1$.
\begin{figure}[h!]
    \centering
    \includegraphics[width=5cm, height=6cm]{Slika5}
    \label{fig:mesh1}
\end{figure}\\
\begin{figure}[h!]
    \centering
    \includegraphics[width=5cm, height=5cm]{Slika6}
    \label{fig:mesh1}
\end{figure}\\


\end{document}
